\begin{surferPage}{Turna}
Bir sözcükte sonsuz sayıda harf...\\
\smallskip
\[y z (x^2	+ y - z)	= 0\]

\vspace{0.3cm}
İzlenimciler evleri ve çayırları binlerce renkli noktacıkla resmettiler. Tıpkı bunun gibi, matematiksel yüzeyler de binlerce noktadan oluşur; bu noktaların ne eni ne kütlesi vardır, sadece denklemi sağlarlar! \\
\vspace{0.3cm}
Sonsuzluğu hayal etmenin bir yolu $1, 2, 3, \dots$ diye saymaya başlamaktır.\\
Her zaman bir sonraki sayı olacaktır. Sonuna kadar saymayı asla başaramayız.\\
\vspace{0.3cm}
Yalnızca yüzeylerde mi sonsuz nokta var? $0$ ve $1$ sayıları arasında da onlardan sonsuz tane var. Olanaksız mı görünüyor? Noktaları sonsuz ufak olarak düşünün; sanki kalınlığı sıfır olan bir kalemle boyanmış gibi... $0$ ile $1$ arasını bunlarla doldurmak için çok fazla nokta işaretlemeniz gerekecek: sonsuz tane!
\end{surferPage}
